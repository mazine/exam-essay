\section{Исполнение программ, разработанных в среде \term{MPS}}
Для того, чтобы код, написанный в среде \MPS{}, можно было запустить, применяются интерпретационный, генерационный или смешанный подходы \cite{semantics}. Так как исходный код в среде \MPS{} храниться непосредственно в виде АСГ, отпадает необходимость в лексическом и синтаксическом разборе кода и процессе разрешения ссылок. То есть исходный код сразу готов к интерпретации. Но интерпретационный подход обладает существенным недостатком: интепретация АСГ не может происходить вне среды \MPS{}, кроме того, интерпретация АСГ как правило менее эффективна, по сравнению с выполнением генерированного машинного кода.

Генерация в среде \MPS{} использует индукционный подход. Базовым в этой индукции является язык, для которого написан текстовый генератор, который преобразует корневые узлы АСГ в текстовые файлы. Например, в среде \MPS{} определен текстовый генератор для языка \term{baseLanguage}, который генерирует код на языке \term{Java}. Текстовые генераторы просты по своей структуре, и фактически реализуют обход АСГ в глубину \cite{cormen} с выводом в файл соответствующего текста для каждого узла дерева.

Вместо текстового генератора для каждого концепта языка можно определить правила преобразования в концепты других языков. Таким образом осуществляется переход индукции. То есть для того, чтобы среда \MPS{} могла преобразовать узел АСГ в текст, для этого узла должен быть либо определен текстовый генератор, либо правила преобразования в узел, который может быть преобразован в текст.

На момент написания данной работы среда \MPS{} способна осуществлять генерацию только моделей целиком, используя при этом следующий алгоритм:
\newcounter{algorithmStep}
\newcommand{\algStep}[1]{\refstepcounter{algorithmStep}\label{alg:#1}}
\newcommand{\refStep}[1]{\ref{alg:#1}}
\begin{enumerate}
 \item \algStep{init}
Установить модель, которую необходимо преобразовать в текст, в качестве \textit{входной модели}.

 \item \algStep{loopStart}
Если ни для одного узла \textit{входной модели} не найдено ни одного правила преобразования в другой узел, перейти к шагу \refStep{textgen}.

 \item \algStep{reduce}
Создать \textit{целевую модель} и скопировать в нее все узлы \textit{входной модели}, при этом,  узлы, для которых найдены правила преобразования, заменить результатом применения этих правил.

 \item \algStep{loopEnd}
Установить полученную \textit{целевую модель} в качестве \textit{входной модели}, перейти к шагу \refStep{loopStart}.

 \item \algStep{textgen}
К каждому корневому узлу \textit{входной модели} применить текстовый генератор.
\end{enumerate}

Выполнению этого алгоритма могут помешать различные ошибки, допущенные при разработке генераторов. Во-первых, правила генерации могут быть написаны таким образом, что выходная модель всегда будет содержать узлы, для которых найдутся правила генерации. Чтобы это не приводило к зависанию генерации, в среде \MPS{} наложено конечное ограничение на количество циклов генерации. То есть, если исходная модель не была преобразована в текст за некоторое конечное количество шагов, то генерация прекращается,  а пользователю выдается сообщение об ошибке.

Во-вторых, для некоторых корневых узлов входной модели, полученной к шагу \refStep{textgen}, может не существовать текстового генератора. Про каждый такой узел пользователю будет выдано сообщение об ошибке, но остальные узлы будут корректно преобразованы в текстовые файлы.

В-третьих, для некорневых узлов на шаге \refStep{textgen} может быть не определено правило преобразования в текст. В этом случае пользователю также будет выдано сообщение об ошибке, а вместо текста, соответствующего такому узлу, в файл будет выведено имя концепта узла.

В-четвертых, при вычислении правила генерации может произойти исключительная ситуация \cite{eckel}. Если это произойдет, генерация будет остановлена, а информация об исключительной ситуации выведена пользователю.
