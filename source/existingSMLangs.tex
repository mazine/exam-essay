\section{Существующие автоматные языки}
Доминирующей в современном программировании является объектно-ориентированная парадигма программирования \cite{meyer}. При
этом для описания поведения объектов, как правило, используется процедурный подход \cite{nepeyvoda}. Однако для целого ряда
задач управления больше подходит автоматный подход \cite{polikarpova}. %\cite{shalyto01,shalyto02}. 
К настоящему моменту существует несколько
языков, поддерживающих программирование автоматов, как в рамках автоматной парадигмы (UniMod FSML \cite{lagunov}, State
Machine \cite{shamgunov}), так и вне ее (SMC \cite{smc}, ТАВР \cite{tsimbaluk}, SCXML \cite{scxml}, Ragel \cite{ragel}).
Каждый из этих языков программирования обладает некоторыми недостатками и ограничениями.

Программирование в среде \term{UniMod} предполагает, что разработка ведется исключительно на базе автоматной парадигмы
\cite{myUMLSwitchEclipse}. Это затрудняет использование среды \term{UniMod} в тех случаях, когда необходимо описать автоматное
поведение класса, в программе, разрабатываемой в рамках традиционного объектно-ориентированного подхода.

Язык \term{State Machine}, представляет собой расширение языка \term{Java}, основанное на паттерне проектирования 
\term{State Machine} \cite{gof}. Этот паттерн позволяет в объектно-ориентированном стиле описывать структуру автомата, однако
такое описание оказывается громоздким, что препятствует использованию языка \term{State Machine} в реальной разработке.

Язык \term{SMC} --- универсальный компилятор автоматов. Этот язык позволяет описывать в автоматы в едином формате и
генерировать целевой код на различных языках общего назначения. Но интеграция этого языка с целевыми языками требует от
разработчиков дополнительных усилий --- необходимо существенно модифицировать код класса для того, чтобы связать его с
автоматом, описывающим его поведение.

В универсальном языке программирования \term{ТАВР} предусмотрены специальные конструкции, облегчающие написание автоматов.
Однако синтаксис этого языка вынуждает разработчика оперировать понятиями более низкоуровневыми, чем "<состояние">. Кроме
того, универсальность языка \term{ТАВР} доставляет дополнительные трудности разработчику, связанные с необходимостью изучать
конструкции нового языка даже для тех задач, для которых подходят более распространенные универсальные языки.

Язык \term{Ragel} предназначен для описания конечных автоматы с помощью регулярных выражений. Такой подход ограничивает 
область применимости этого языка задачами лексического анализа и спецификации протоколов,

Стандарт языка \term{SCXML} --- спецификация XML-представления автоматов Харела. %\cite{harel}. 
Использовать этот язык 
непосредственно для написания кода автоматов на нем крайне затруднительно, скорее он подходит для обмена автоматами Харела
между приложениям.

Существуют и другие языки так или иначе позволяющие использовать автоматы при программировании. Что доказывает популярность
автоматного подхода для решения различных задач программирования. В связи с этим, процесс создания языка в среде \MPS{}
будет продемонстрирован на примере языка \term{jet\-bra\-ins.mps.ba\-seLan\-gu\-age.sta\-teMa\-chi\-ne}, являющегося автоматным расширением
языка \term{baseLanguage}.
