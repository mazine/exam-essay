\section{Поддержка языка \term{Java} в среде \MPS{}}
Ядро среды \MPS{} написано на кросс"=платформенном языке \term{Java} \cite{eckel}. В связи с этим в среде \MPS{} существуют развитые средства для взаимодействия с Java"=платформой. Для написания Java"=кода в среде \MPS{} разработан язык \term{jet\-bra\-ins.mps.ba\-seLan\-gu\-age} (далее \term{baseLanguage}). Этот язык является почти полной реализацией спецификации \term{Java 5} \cite{java5spec}. В нем определены такие концепты, как "<класс">, "<интерфейс">, "<метод">, "<предложение">, "<выражение"> и т.д.

Библиотеки, написанные непосредственно на языке \term{Java}, могут быть использованы при разработке кода на языке \term{baseLanguage}. Для этого среда \MPS{} предоставляет доступ к пакетам библиотек, как к моделям. Классы, поля и методы из этих пакетов представляются экземплярами соответствующих концептов языка \term{baseLanguage}.

Таким образом, для того, чтобы в своей модели написать на языке \term{baseLanguage} класс, выводящий на консоль сообщение "<Hello, world!">, необходимо добавить в набор, используемых моделью языков --- \term{baseLanguage}, а в набор, импортированных моделей ---
\term{java.lang@java\_stub} и \term{java.io@java\_stub}
(Рис. \ref{fig:Import}). Суффикс \term{@java\_stub} в именах моделей, указывает на то, что эти модели являются на самом деле пакетами языка \term{Java}.
\begin{figure}
 \centering
 \fbox{
  \includegraphics[width=0.95\textwidth]{Import.png}
 }
 \caption{Код класса, выводящего на консоль строку «Hello, world!», и список используемых языков и импортированных моделей}
 \label{fig:Import}
\end{figure}
