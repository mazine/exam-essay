\section{Актуальность проблемно-ориентированных языков}
Продукт, который выдает программист, не код, а решение задачи \cite{dmitriev}. При этом предпочтительно более компактное, эффективное и понятное решение. В то же время исходный код программы все чаще воспринимается не столько как описание решения, но как средство взаимодействия между разработчиками. Это связано с тем, что программные продукты, в отличие от других инженерных разработок, часто инновационны \cite{brooks}, требования к инновационным продуктам тяжело формализуемы, поэтому заказчик редко представляет себе с самого начала, чего он хочет, поэтому популярна эволюционная модель разработки \cite{gost12207-99,evolutionModel}, поэтому исходный код часто подвергается значительным изменениям разными разработчиками [ссылка на совместное владение кодом].

В связи с этим в настоящее время все большее распространение получают проблемно-ориентированные языки программирования \cite{fowler01}. Такие языки позволяют описывать решения задач тех областей, для которых они предназначены, более выразительно и компактно по сравнению с универсальными языками.

Когда разработчик, в процессе неформального общения, описывает свое решение другому разработчику, он оперирует абстракциями более высокого уровня по сравнению с теми, что используются при программировании на универсальном языке. Например, описывая элемент пользовательского интерфейса, разработчик скажет, что в правом верхнем углу окна будет расположено поле для ввода текста, а не "<получить ссылку на объект окна, добавить в верхний правый угол текстовое поле ввода, установить текущее значение поля ввода, и т.д.">. Недостатком неформального описания, является неполнота и неточность описания решения. Действительно, при неформальном описании никак не описывается положение текстового поля при изменении размеров окна, остается не определенным, в какой момент следует обрабатывать пользовательский ввод и т.п. Хороший проблемно-ориентированный язык программирования должен позволять описывать решения в терминах абстракций, которыми оперирует разработчик, но при этом предоставлять возможность описывать решения достаточно точно и полно.
