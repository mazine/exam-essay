\section{Язык \term{jet\-bra\-ins.mps.ba\-se\-Lan\-gu\-age.sta\-te\-Ma\-ch\-ine}}
Проблемно"=ориентированный язык \term{jet\-bra\-ins.mps.ba\-se\-Lan\-gu\-age.sta\-te\-Ma\-chi\-ne} (далее \term{stateMachine})
представляет собой автоматное расширение языка \term{baseLanguage}. Основная цель, которая приследовалась при разработке
языка \term{stateMachine}, --- создать средство, позволяющее описывать поведение классов в виде автоматов, не накладывая на
сами классы никаких дополнительных ограничений. Более того, автоматное описание поведения класса должно быть
инкапсулировано. То есть код, использующий класс не должен "<знать"> о том, каким образом задано поведение класса. Этот
подход отличается от предложенного в работе \cite{myUMLSwitchEclipse} тем, что позволяет использовать автоматы не только в
программах написанных в соответствии со SWITCH-технологией, но и в традиционных объектно-ориентированных программах.

В качестве примера использования языка \term{stateMachine} взята система управления лифтом \cite{knuth}. Используемая
реализация системы является упрощенным решением задачи управления лифтом, описанном в работе \cite{naumov}. Логика
управления всеми подсистемами лифта реализована в одном автомате.

Каждый автомат в языке \term{stateMachine} связан с некторым классом и описывает его поведение. Для удобства редактирования
автомат описывается в отдельном корневом узле \pic{\ref{fig:ElevatorStateMachine}}.

\begin{figure}
 \centering
 \fbox{
  \includegraphics[width=0.7\textwidth]{ElevatorStateMachine.png}
 }
 \caption{Фрагмент автомата управляющего лифтом}
 \label{fig:ElevatorStateMachine}
\end{figure}

Для того чтобы задать поведение класса с помощью автомата в языке \term{stateMachine} необходимо в этом классе определить
события, на которые будет реагировать автомат. Особенностью языка \term{stateMachine} является то, что события в нем --- это
методы специального вида. В языке \term{Java} декларации методов различаются по способу реализации:
\begin{itemize}
 \item обычные методы, реализация которых следует сразу за объявлением метода;
 \item нативные методы, реализованные на платформо-зависимых языках программирования;
 \item абстрактные методы, вообще не имеющие реализации.
\end{itemize}
Событие в языке \term{stateMachine} --- это метод, реализация которого находится в автомате и зависит от текущего состояния
автомата. Например, в классе \term{Elevator} объявлены следующие события \pic{\ref{fig:ElevatorEvents}}
\begin{description}
 \item[doorsOpened, doorsClosed] --- события, которые посылает объект управления \term{DoorsEngine}, когда двери оказываются в максимально открытом и полностью закрытом положении соответственно;
 \item[floorReached] --- событие, которое посылает объект управления \term{ElevatorEngine}, когда лифт достигает очередного этажа;
 \item[call] --- событие, которое получает автомат, когда пассажир нажимает кнопку вызова лифта на этаже, или в кабине лифта;
 \item[openDoors] --- событие, которое получает автомат, когда пассажир нажимает кнопку экстренного открывания дверей в кабине лифта;
 \item[loadingTimeout] --- событие, которое посылает объект управления \term{LoadingTimer}, когда истекает время ожидания погрузки пассажиров.
\end{description}
\begin{figure}
 \centering
 \fbox{
  \includegraphics[width=0.9\textwidth]{ElevatorEvents.png}
 }
 \caption{События автомата управляющего лифтом}
 \label{fig:ElevatorEvents}
\end{figure}
Так как события являются обычными методами, их можно использовать в качестве реализации абстрактных методов. Это позволяет
уменьшить количество зависимостей в программе. Например, объект управления \term{ElevatorEngine} не имеет непосредственных
ссылок на класс \term{Elevator}. Вместо этого в программе объявлен интерфейс \term{ElevatorEngineListener}, и объект
управления \term{ElevatorEngine} извещает о событиях экземпляры этого интерфейса. Класс \term{Elevator} реализует интерфейс
\term{ElevatorEngineListener} и поэтому может обрабатывать события объекта управления \term{ElevatorEngine}. Таким образом
удается классически реализовать паттерн проектирования \term{Обозреватель} \cite{gof}.

Реакция автомата на событие зависит от состояния, в котором автомат находится. Состояния автомата бывают двух типов: обычные
и конечные. Конечные состояния отличаются тем, что не могут иметь исходящих переходов. То есть, когда автомат оказывается в
конечном состоянии он перестает обрабатывать события.

Первое состояние, объявленное в автомате, считается начальным. При создании экземпляра класса, для которого определен
автомат, после выполнения конструктора осуществляется переход в начальное состояние.

Обычные состояния могут быть вложены друг в друга. При этом, если состояние содержит

События, как и другие методы, могут иметь формальные параметры. Значения этих параметров доступны в условиях и действиях на
переходах.



Каждый автомат в языке \term{stateMachine} связан с некоторым классом, имеет доступ к полям и может вызывать методы этого
класса. Поэтому в языке \term{stateMachine} нет необходимости в специальных конструкциях для взаимодействия между
автоматами, так как вместо вложения одного автомата в другой можно использовать агрегацию одного класса с автоматом, другим
классом с автоматом, а посылка события из одного автомата в другой ни что иное как вызов метода.